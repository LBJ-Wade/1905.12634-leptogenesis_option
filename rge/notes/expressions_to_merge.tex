\documentclass[a4paper,11pt]{scrartcl}
\pdfoutput=1
\bibliographystyle{utphys27mod}

% ----------------------------------------------------------- Packages
\usepackage{amsmath,amssymb,url,cite,slashed,cancel,booktabs,hyperref,graphicx,xspace,subcaption}
%%%UNUSED%%% \usepackage{feynmp,enumerate,multirow,wrapfig}
\renewcommand\citepunct{,\penalty1000\hskip.13emplus.1emminus.1em\relax} % no line-break in \cite
\renewcommand\thefootnote{*\arabic{footnote}}
\numberwithin{equation}{section}

% MATH NOTATION
\newcommand\w[1]{_{\mathrm{#1}}}
\newcommand\vc[1]{{\boldsymbol{#1}}}
\newcommand\dd{\mathop{}\!\mathrm{d}}
\newcommand\DD{\mathop{}\!\mathrm{D}}
\newcommand\ee{\mathop{}\!\mathrm{e}}
\newcommand\abs[1]{\lvert#1\rvert}
\newcommand\norm[1]{\lVert#1\rVert}
\newcommand\Abs[1]{\left\lvert#1\right\rvert}
\newcommand\Norm[1]{\left\lVert#1\right\rVert}

\newcommand\ii{\mathrm{i}}
\newcommand\co[1]{\mathrm{c}_{#1}}
\newcommand\si[1]{\mathrm{s}_{#1}}

\newcommand\pmat[1]{\begin{pmatrix}#1\end{pmatrix}}
\DeclareMathOperator{\Order}{\mathcal{O}}
\DeclareMathOperator{\Tr}{\mathrm{Tr}}

\newcommand\oneone{1}
\newcommand{\dn}[3]{\frac{\dd^#1 #2}{\dd #3^#1}}    % derivatives
\newcommand{\pdn}[3]{\frac{\partial^#1 #2}{\partial #3^#1}}
\newcommand{\pd}[2]{\frac{\partial #1}{\partial #2}}
\newcommand\paren[3]{\def\temp{#3}\Bigl(\frac{#1}{#2}\Bigr)\ifx\oneone\temp\relax\relax\else^{#3}\fi}
\newcommand\vev[1]{\langle#1\rangle}
\newcommand{\mean}[1]{\left\langle #1 \right\rangle}

\newcommand\hc{\text{h.c.}}


% units
\newcommand\unit[1]{\,\mathrm{#1}\xspace}
\newcommand\eV{\unit{eV}}
\newcommand\keV{\unit{keV}}
\newcommand\MeV{\unit{MeV}}
\newcommand\GeV{\unit{GeV}}
\newcommand\TeV{\unit{TeV}}
\newcommand\PeV{\unit{PeV}}
\newcommand\fb{\unit{fb}}
\newcommand\pb{\unit{pb}}
\newcommand\iab{\unit{ab^{-1}}}
\newcommand\ifb{\unit{fb^{-1}}}
\newcommand\ipb{\unit{pb^{-1}}}
\newcommand\fm{\unit{fm}}


% scientific form of numbers
\makeatletter
\def\EE{\@ifnextchar-{\@@EE}{\@EE}}
\def\@EE#1{\ifnum#1=1 \times\!10 \else \times\!10^{#1}\fi}
\def\@@EE#1#2{\times\!10^{-#2}}
\makeatother





%%%%%%%%%% REMOVE THIS BLOCK WHEN FINALIZE %%%%%%%%%%
%\usepackage{fancyhdr,xcolor}
%\usepackage[hhmmss]{datetime}
%\newdateformat{mydate}{\THEDAY\;\shortmonthname.\;\THEYEAR}
%\renewcommand{\headrulewidth}{0pt}
%\makeatletter
%\gdef\@fpheader{\color{gray}{\textsc{Draft \jobname\ compiled at \mydate\today~\currenttime}}}
%\lhead{}
%\rhead{\footnotesize\@fpheader}
%\makeatother
\usepackage{scrlayer-scrpage,color,soul}
\usepackage[hhmmss]{datetime}
\newdateformat{mydate}{\THEDAY\;\shortmonthname.\;\THEYEAR}
\addtokomafont{pagehead}{\small\normalfont}
\ohead{\texttt{[\jobname~@~\mydate\today~\currenttime]}}
\newcommand{\comment}[1]{{\textbf{\small \color{red} [#1]}}}
\newcommand{\rem}[1]{{\textbf{\small \color{red} $<$#1$>$}}}
\newcommand{\TODO}[1]{{\textbf{\small \color{red} [TODO: #1]}}}
\newcommand{\TOWRITE}[1]{{{\footnotesize \color{blue} (#1)}}}
\newcommand{\cmark}{\ding{51}} % check mark
\newcommand{\xmark}{\ding{55}} % X mark
\newcommand{\C}{\color{magenta}}
%%%%%%%%%%%%%%%%%%%%%%%%%%%%%%%%%%%%%%%%%%%%%%%%%%%%%



\newcommand\YN{y}
\newcommand\NR{{N\w R}}
\newcommand\NRbar{{\overline N\w R}}
\newcommand\NRc{{N\w R^{\mathrm c}}}


\newcommand{\trans}{^{\mathrm T}}
\newcommand{\spmat}[1]{\left(\begin{smallmatrix}#1\end{smallmatrix}\right)}
\DeclareMathOperator{\diag}{\mathrm{diag}}

%----------------------------------------------------------------------------------------

\title{Expressions to merge}

\begin{document}
\section{set up}
\subsection{Lagrangian}
Following the notation [1611.03827] by Bhupal et al.\ and without specifying the basis,
\begin{equation}
 -\mathcal L \supset \YN_{ak} \NRbar_a \tilde\phi^\dagger L_k + \frac12 \NRbar_a M_{ab} \NRc_b + \text{h.c.}
\\
\end{equation}

$\phi = (\phi^+, \phi^0)\trans$

$\tilde\phi=\ii\sigma_2\phi^* = \spmat{0&1\\-1&0}\spmat{\phi^-\\\phi^{0*}} = \spmat{\phi^{0*}\\-\phi^-}$

$\tilde\phi^\dagger=(\ii\sigma_2\phi^*)^\dagger = \pmat{\phi_0 & -\phi^+}$


\begin{align}
  -\mathcal L &\supset \YN_{ak} \NRbar_a (\phi^{0}\nu_k-\phi^+l_k) + \frac12 \NRbar_a M_{ab} \NRc_b + \text{h.c.}
\end{align}




Neutrino mass matrix is better shown in two-component $\NR=\spmat{0\\n^\dagger}$:
\begin{align}
  -\mathcal L &
\supset \vev{\phi_0}(\YN)_{ai} n_a\nu_i + \frac12 M_{ab}n_an_b + \text{h.c.}\\
&=
\frac12\pmat{\nu_i & n_a}
\pmat{
 0_{ij} & \vev{\phi_0}(\YN)_{bi} \\
 \vev{\phi_0}(\YN)_{aj} & M_{ab}
}
\pmat{\nu_j \\ n_b} + \text{h.c.}
\end{align}
or we will write down, assuming the notation is understood,
\begin{equation}
 M_\nu = \pmat{
   0 & \vev{\phi^0}\YN\trans\\
   \vev{\phi^0}\YN & M
}
\end{equation}
and perform Autonne--Takagi diagonalization:
\begin{equation}
 U_0\trans M_\nu U_0 = \diag(m_1, m_2, m_3, M_1, M_2)
\end{equation}


\subsection{Casas--Ibarra parameterization}
We split this Autonne--Takagi diagonalization procedure to two steps:
\begin{align}
   &U_1\trans  M_\nu U_1 = \spmat{A& 0 \\ 0&B},\\
   &\spmat{U_2&0\\0&U_3}\trans U_1\trans M_\nu U_1 \spmat{U_2&0\\0&U_3} = \spmat{m\w{diag} & 0 \\ 0 & M\w{diag}},
\end{align}
where $m\w{diag}=\diag(m_1, m_2, m_3)$ and $M\w{diag}=\diag(M_1, M_2)$.
The result of the first step is well-known in series-expanded form:
\begin{align}
U_1 &\simeq \pmat{
  1 - \frac{\vev{\phi_0}^2}{2}\YN^\dagger (MM^*)^{-1}\YN &
 \vev{\phi_0} \YN^\dagger M^{*-1}
\\
-\vev{\phi_0} M^{-1}\YN &
  1 - \frac{\vev{\phi_0}^2}{2} M^{-1}\YN\YN^\dagger M^{*-1}
},\\
 A &\simeq -\vev{\phi_0}^2 \YN\trans M^{-1} \YN,\\
 B &\simeq M + \frac{\vev{\phi_0}^2}{2} \left( \YN\YN^\dagger M^{*-1} + M^{*-1} \YN^* \YN\trans\right)
\end{align}
The second step is expressed by
\begin{align}
 U_2\trans A U_2 &= m\w{diag}, &
 U_3\trans B U_3 &= M\w{diag}.
\end{align}
We also have the expression of the mass eigenstates:
\begin{align}
 \text{lighter}:&\quad
  U_2^\dagger\left[
\nu
-\vev{\phi_0} \YN^\dagger M^{*-1} n
-\frac{\vev{\phi_0}^2}{2} \YN^\dagger (MM^*)^{-1}\YN\nu
\right]\\
 \text{heavier}:&\quad
  U_3^\dagger\left[
n
+\vev{\phi_0} M^{-1}\YN\nu
-\frac{\vev{\phi_0}^2}{2} M^{-1}\YN\YN^\dagger M^{*-1}n
\right]\\
% R'^\dagger(U_{12}^\dagger \nu + U_{22}^\dagger n).
\end{align}
Combining them,
\begin{align}
 m\w{diag} &=  U_2\trans A U_2\\
&= -\vev{\phi_0}^2U_2\trans \YN\trans M^{-1} \YN U_2 + \Order(M\epsilon^4)\\
&=-\vev{\phi_0}^2U_2\trans \YN\trans B^{-1} \YN U_2 + \Order(M\epsilon^4)\\
&=-\vev{\phi_0}^2U_2\trans \YN\trans U_3 M\w{diag}^{-1}U_3\trans \YN U_2 + \Order(M\epsilon^4)\\
&=: R'{}\trans R' + \Order(M\epsilon^4)
\end{align}
with $\Order(M\epsilon^n)\sim\Order(\vev{\phi_0}^n/M_1^{n-1})$ and $R':=-\ii\vev{\phi_0}M\w{diag}^{-1/2}U_3\trans yU_2$. Now, the Yukawa coupling is given by
$y=\ii\vev{\phi_0}^{-1}U_3^*M\w{diag}^{1/2}R'U_2^\dagger$.



Here, for two heavy neutrino scenarios, $R'$ can be parameterized as
\begin{align}
R'\w{NH} &= \begin{pmatrix}
0 & +\sqrt{m_2}\cos z  & \zeta\sqrt{m_3}\sin z\\
0 & -\sqrt{m_2}\sin{z} & \zeta\sqrt{m_3}\cos z
\end{pmatrix}
=
\begin{pmatrix}
0 & +\cos z  & \zeta\sin z\\
0 & -\sin{z} & \zeta\cos z
\end{pmatrix}\sqrt{m\w{diag}},
\\
R'\w{IH} &= \begin{pmatrix}
+\sqrt{m_1}\cos z  & \zeta\sqrt{m_2}\sin z & 0\\
-\sqrt{m_1}\sin{z} & \zeta\sqrt{m_2}\cos z & 0
\end{pmatrix}
=
\begin{pmatrix}
+\cos z  & \zeta\sin z & 0\\
-\sin{z} & \zeta\cos z & 0
\end{pmatrix}\sqrt{m\w{diag}}
\end{align}
for normal hierarchy (NH; $m_1=0<m_2<m_3$) and inverted hierarchy (IH; $m_3=0<m_1<m_2$) cases.
Therefore, the Yukawa couplings are given by
\begin{align}
 \YN = \ii\vev{\phi_0}^{-1}U_3^* \sqrt{M\w{diag}}R\sqrt{m\w{diag}}U_2^\dagger,
\end{align}
which is the Casas--Ibarra parameterization in general basis; $R$ is given by
\begin{align}
R\w{NH} &= \begin{pmatrix}
0 & +\cos z  & \zeta\sin z\\
0 & -\sin{z} & \zeta\cos z
\end{pmatrix},
&
R\w{IH} &=
\begin{pmatrix}
+\cos z  & \zeta\sin z & 0\\
-\sin{z} & \zeta\cos z & 0
\end{pmatrix}.
\end{align}

We have not yet defined the lepton basis.
We can assume that we have been using, from the beginning, the charged lepton mass basis for $L$.
Then, we identify the PMNS matrix ($U_{li}$ in Eq.~(14.1) of PDG2018, where $i$ for mass and $l$ for gauge indices) as
\begin{equation}
 U\w{PMNS} \simeq U_2.
\end{equation}
Similarly, the basis for $\NR$ is such that $U_3 \simeq 1$, which corresponds to
\begin{equation}
 M = B + \Order(\epsilon^2) = U_3^* M\w{diag} U_3^\dagger +\Order(\epsilon^2) = M\w{diag} + \Order(\epsilon^2),
\end{equation}
i.e., the basis in which $M_{ab}\simeq\diag(M_1, M_2)$ with $0<M_1\le M_2$.
These basis choice gives the well-known Casas--Ibarra parameterization,\footnote{
If we took the basis in which $M_{ab}\simeq\diag(-M_1, -M_2)$, then $U_3=\diag(-i, -i)$ and we can remove $i$, but it is less plausible and we put $i$ in the parameterization.
}
\begin{equation}
 y = \ii\vev{\phi_0}^{-1}\sqrt{M\w{diag}}R\sqrt{m\w{diag}}U\w{PMNS}^\dagger,
\end{equation}
where
\begin{equation}\begin{split}
 U\w{PMNS} &= \pmat{
                 c_{12}      c_{13}             &                 s_{12}      c_{13}            &       s_{13} e^{-i\delta} \\
 -s_{12}c_{23} - c_{12}s_{23}s_{13}e^{i\delta}  & +c_{12}c_{23} - s_{12}s_{23}s_{13}e^{i\delta} & s_{23}c_{13}\\
 +s_{12}s_{23} - c_{12}c_{23}s_{13}e^{i\delta}  & -c_{12}s_{23} - s_{12}c_{23}s_{13}e^{i\delta} & c_{23}c_{13}
}
\\&\qquad\qquad
\times\diag(1, e^{i\alpha_{21}/2}, e^{i\alpha_{31}/2}).
\end{split}\end{equation}

\subsection{Higgs potential}
Here we calculate the threshold correction to the Higgs potential from the consistency of the one-loop effective potential $V\w{eff}$.
The theory with and without the right-handed neutrinos, which we call the ``full'' theory and the EFT, should have the same effective potential at some matching scale $Q_0$.
Hence, comparing the effective potential, we can derive the corrections for the tree-level Higgs potential.

Following [1611.03827], we denote the tree-level Higgs potential by
\begin{equation}
 V = -\mu^2|\phi|^2 + \lambda|\phi|^4,
\end{equation}
which gives $m_h^2 = 2\mu^2 = 2\lambda v^2$ with $\vev{\phi}=v/\sqrt{2}$.
The effective potential is given by, at the one-loop level,
\begin{equation}
 V\w{eff}(Q) = V(Q) + V^{(1)}(Q),
\end{equation}
and the difference between the two theories are
\begin{equation}
\Delta V := V\w{full}^{(1)}(\phi;Q) - V\w{EFT}^{(1)}(\phi;Q)
= \sum_{a=1,2}\frac{-2}{64\pi^2}M_a(\phi)^4\left(\log\frac{M_a(\phi)^2}{Q^2}-\frac32\right).
\end{equation}
Therefore, at the matching scale $Q_0$, the tree-level potential should satisfy
\begin{equation}
 V\w{full}(Q_0) - V\w{EFT}(Q_0) = -\Delta V,
\end{equation}
which gives the threshold corrections.

The difference $\Delta V$ is expanded in terms of $\phi$ as
\begin{align}
 \Delta V
&= \text{(const.)} - \Delta \mu^2 |\phi|^2 + \Delta \lambda |\phi|^4 + \Order\left(|\phi|^6\right),
\end{align}
where
\TODO{Now $M_a$ is the diagonal majorana masses...}
\begin{align}
 &\Delta\mu^2= -\sum_{a=1,2}\frac{H_a}{8\pi^2}M_a^2\left(1-\log\frac{M_a^2}{Q^2}\right),
\\
 &\Delta\lambda
=
-\frac{1}{16\pi^2}
\Bigl[f_1 \Tr(Y Y^\dagger Y^*Y\trans)
+f_2\Tr(YY^\dagger YY^\dagger)
+f_3H_1^2+f_4H_2^2
\Bigr];
\end{align}
the coefficients are $H_1 = (YY^\dagger)_{11}$, $H_2 = (YY^\dagger)_{22}$, and
\begin{align*}
 f_1&=\frac{2M_1 M_2}{M_2^2-M_1^2}\log\frac{M_2}{M_1},&
 f_2&=\frac{M_2^2\log(M_2^2/Q^2) - M_1^2\log(M_1^2/Q^2)}{M_2^2 - M_1^2} - 1,\\
 f_3 &= 2-\frac{2M_2\log(M_2/M_1)}{M_2-M_1},&
 f_4 &= 2-\frac{2M_1\log(M_2/M_1)}{M_2-M_1};
\end{align*}
for $M_2\simeq M_1$, they approach to $f_1=1$, $f_2=\log({M_1^2}/{Q^2})$, and $f_3=f_4=0$, which gives
\begin{align}
 \Delta \mu^2
&\simeq -\frac{M_1^2}{8\pi^2}\Tr(YY^\dagger)\left(1-\log\frac{M_1^2}{Q^2}\right),\\
 \Delta \lambda
&\simeq
-\frac{1}{16\pi^2}\Bigl[
 \Tr(Y Y^\dagger Y^*Y\trans) + \Tr(YY^\dagger YY^\dagger)\log\frac{M_1^2}{Q^2}
\Bigr].
\end{align}

The neutrino-option scenario requires $\mu^2\w{full}=0$, i.e.,
\begin{equation}
 \mu^2\w{full}(Q_0) = \mu^2\w{EFT}(Q_0) - \Delta \mu^2(Q_0) = 0.
\end{equation}
Therefore, the condition for the neutrino-option scenario is summarized as
\begin{equation}
 \mu^2\w{EFT}(Q_0) = -\frac{M_1^2}{8\pi^2}\Tr(YY^\dagger)\left(1-\log\frac{M_1^2}{Q_0^2}\right)
\end{equation}
at the one-loop level, or using $Q_0=M_1\ee^{-3/4}$,
\begin{equation}
 \mu^2\w{EFT}(Q_0) = +\frac{M_1^2}{16\pi^2}\Tr(YY^\dagger).
\end{equation}




\appendix
\section{Comparison}
Here we compare our results with literature; we use {\C magenta color} for the variables/notations in other papers, while black letters are in our notation
\begin{align*}
 V &= -\mu^2(\phi^\dagger \phi) + \lambda(\phi^\dagger \phi)^2,&
m_h^2 &= 2\mu^2=2\lambda v^2,&
 v &:= \sqrt2 \vev{\phi},
\end{align*}
or $v\sim246\GeV$, $\mu^2\sim(88\GeV)^2\sim7700\GeV^2$, and $\lambda\sim0.13$.

Brivio--Trott~\ref{1809.03450} uses
\begin{equation}
\C
 V(H^\dagger H) = -\frac{m_{0}^{2}+\Delta m^{2}}{2}\left(H^{\dagger} H\right)+\left(\lambda_{0}+\Delta \lambda\right)\left(H^{\dagger} H\right)^{2} + \cdots
\end{equation}
and derives
\begin{equation}
\C
\Delta m^{2}=-\frac{\left|\omega_{p}\right|^{2} M_{p}^{2}}{4 \pi^{2}}\left(1+\log \frac{\mu^{2}}{M_{p}^{2}}\right)=\frac{1}{8 \pi^{2}}\left[M_{1}^{2}\left|\omega_{1}\right|^{2}+M_{2}^{2}\left|\omega_{2}\right|^{2}\right],
\end{equation}
which is consistent with our results as $\Delta \mu^2 = \C\Delta m^2/2$.\footnote{Their $\Delta \lambda$, which SI thinks incorrect, is totally inconsistent with our results.}


Casas et al.~\ref{hep-ph/9904295} uses effective potential approach:
\begin{align}
 \C
V=V\w{SM} + \Delta V_\nu,\qquad
V\w{tree}=-\frac12 m^2\phi^2+\frac18\lambda\phi^2+\Omega,
\end{align}
where ${\C v=\vev{\phi}}=246\GeV=v=\sqrt{2}\vev{\phi}$, so (fixing a typo)
$  {\C V\w{tree}}=-{\C m^2}\phi^2+{\C (\lambda/2)}\phi^4 $.
Then,
\begin{equation}\C
 \Delta V_{\nu}=-\frac{1}{32 \pi^{2}}\left[m_{\nu_{1}}^{4} \log \frac{m_{\nu_{1}}^{2}}{\mu^{2}}+
m_{\nu_{2}}^{4} \log \frac{m_{\nu_{2}}^{2}}{\mu^{2}}\theta(\mu-M) \right]
\end{equation}
and
\begin{equation}\C
 \Delta\w{th} V=-\frac{2}{64 \pi^{2}}m_{\nu_{2}}^{4} \log \frac{m_{\nu_{2}}^{2}}{\mu^{2}}=\Delta V_{\mu>M} - \Delta V_{\mu<M},
\end{equation}
where one should note that ${\C \mu^2}=Q^2\ee^{3/2}$.
Their results, which is evaluated at $\C\mu=M$,
\begin{equation}\C
 \Delta\w{th} m^{2}=\frac{1}{16 \pi^{2}} Y_{\nu}^{2} M^{2},
\qquad
\Delta\w{th} \lambda=-\frac{5}{16 \pi^{2}} Y_{\nu}^{4},
\end{equation}
are consistent with our results; note that
${\C\Delta\w{th}m^2} = \Delta\mu^2$ and ${\C\Delta\w{th}\lambda/2}=\Delta\lambda$.


\end{document}

