\documentclass[english,11pt,a4paper]{article}
\pdfoutput=1
\bibliographystyle{utphys27mod}

\usepackage{amsmath,amssymb,url,slashed,xstring,cancel,booktabs,hyperref,graphicx,xspace,subcaption}
\numberwithin{equation}{section}  % section number in equation label



% MATH NOTATION
\newcommand\w[1]{_{\mathrm{#1}}}
\newcommand\vc[1]{{\boldsymbol{#1}}}
\newcommand\dd{\mathop{}\!\mathrm{d}}
%\newcommand\DD{\mathop{}\!\mathrm{D}} % not using roman D, e, and i
%\newcommand\ee{\mathop{}\!\mathrm{e}}
%\newcommand\ii{\mathrm{i}}
\newcommand\pmat[1]{\begin{pmatrix}#1\end{pmatrix}} % matrix with parentheses borders

\DeclareMathOperator{\Order}{\mathcal{O}}
\DeclareMathOperator{\Tr}{\mathrm{Tr}}

\def\oneone{1}
\newcommand{\pd}{\partal}
\newcommand{\dn}[3]{\frac{\dd^#1 #2}{\dd #3^#1}}    % derivatives
\newcommand{\pdn}[3]{\frac{\partial^#1 #2}{\partial #3^#1}}
\newcommand\paren[3]{\def\temp{#3}\Bigl(\frac{#1}{#2}\Bigr)\ifx\oneone\temp\relax\relax\else^{#3}\fi}
\newcommand{\bra}[1]{\langle #1 |}  % consider using braket.sty if you write compilcated exp
\newcommand{\ket}[1]{| #1 \rangle}
\newcommand\vev[1]{\langle#1\rangle}
\newcommand{\mean}[1]{\left\langle #1 \right\rangle}

\newcommand\hc{\text{h.c.}}


% units
\newcommand\unit[1]{~\mathrm{#1}\xspace}
\newcommand\eV{\unit{eV}}
\newcommand\keV{\unit{keV}}
\newcommand\MeV{\unit{MeV}}
\newcommand\GeV{\unit{GeV}}
\newcommand\TeV{\unit{TeV}}
\newcommand\PeV{\unit{PeV}}
\newcommand\fb{\unit{fb}}
\newcommand\pb{\unit{pb}}
\newcommand\iab{\unit{ab^{-1}}}
\newcommand\ifb{\unit{fb^{-1}}}
\newcommand\ipb{\unit{pb^{-1}}}
\newcommand\fm{\unit{fm}}


% scientific form of numbers
\makeatletter
\def\EE{\@ifnextchar-{\@@EE}{\@EE}}
\def\@EE#1{\ifnum#1=1 \times\!10 \else \times\!10^{#1}\fi}
\def\@@EE#1#2{\times\!10^{-#2}}
\makeatother





%%%%%%%%%% REMOVE THIS BLOCK WHEN FINALIZE %%%%%%%%%%
\usepackage{fancyhdr,xcolor}
\usepackage[hhmmss]{datetime}
\newdateformat{mydate}{\THEDAY\;\shortmonthname.\;\THEYEAR}
\renewcommand{\headrulewidth}{0pt}
\makeatletter
\gdef\@fpheader{\color{gray}{\textsc{Draft \jobname\ compiled at \mydate\today~\currenttime}}}
\lhead{}
\rhead{\footnotesize\@fpheader}
\makeatother
\newcommand{\comment}[1]{{\textbf{\small \color{red} [#1]}}}
\newcommand{\rem}[1]{{\textbf{\small \color{red} $<$#1$>$}}}
\newcommand{\TODO}[1]{{\textbf{\small \color{red} [TODO: #1]}}}
\newcommand{\TOWRITE}[1]{{{\footnotesize \color{blue} (#1)}}}
\newcommand{\cmark}{\ding{51}} % check mark
\newcommand{\xmark}{\ding{55}} % X mark
%%%%%%%%%%%%%%%%%%%%%%%%%%%%%%%%%%%%%%%%%%%%%%%%%%%%%



\newcommand\YN{Y_\nu}
\newcommand\NR{{N\w R}}
\newcommand\NRbar{{\overline N\w R}}
\newcommand\NRc{{N\w R^{\mathrm c}}}


\newcommand{\trans}{^{\mathrm T}}
\newcommand{\spmat}[1]{\left(\begin{smallmatrix}#1\end{smallmatrix}\right)}
\DeclareMathOperator{\diag}{\mathrm{diag}}

%----------------------------------------------------------------------------------------

\title{Expressions to merge}

\begin{document}
\section{set up}
\subsection{Lagrangian}
A wise choice would be to follow the notation [1611.03827] by Bhupal et al.

\begin{equation}
 -\mathcal L \supset (\YN)_{ai} \NRbar_a \tilde\phi^\dagger L_i + \frac12 \NRbar_a M_{ab} \NRc_b + \text{h.c.}
\\
\end{equation}

$\phi = (\phi^+, \phi^0)\trans$

$\tilde\phi = \spmat{0&1\\-1&0}\spmat{\phi^-\\\phi^{0*}} = \spmat{\phi^{0*}\\-\phi^-}$

\begin{align}
  -\mathcal L &\supset (\YN)_{ai} \NRbar_a (\phi^{0}\nu_i-\phi^+l_i) + \frac12 \NRbar_a M_{ab} \NRc_b + \text{h.c.}\\
   &= \epsilon_{AB}(\YN)_{ai} \NRbar_a L_{iA}\phi_B + \frac12 \NRbar_a M_{ab} \NRc_b + \text{h.c.}
\end{align}

This Lagrangian is written without specifying the basis.
For CI parameterization we first define $M_a$ and $m_i$ by the physical mass of $N_a$ and $\nu_i$, where $M_1\le M_2$ but the basis for $\nu_i$ is ``as usual''.


Neutrino mass matrix is better shown in two-component $\NR=\spmat{0\\n^\dagger}$:
\begin{align}
  -\mathcal L &
\supset \vev{\phi_0}(\YN)_{ai} n_a\nu_i + \frac12 M_{ab}n_an_b + \text{h.c.}\\
&=
\frac12\pmat{\nu_i & n_a}
\pmat{
 0_{ij} & \vev{\phi_0}(\YN)_{bi} \\
 \vev{\phi_0}(\YN)_{aj} & M_{ab}
}
\pmat{\nu_j \\ n_b} + \text{h.c.}
\end{align}
or we will write down, assuming the notation is understood,
\begin{equation}
 M_\nu = \pmat{
   0 & \vev{\phi^0}\YN\trans\\
   \vev{\phi^0}\YN & M
}
\end{equation}
and perform Autonne--Takagi diagonalization:
\begin{equation}
 U_0\trans M_\nu U_0 = \diag(m_1, m_2, m_3, M_1, M_2)
\end{equation}


\subsection{Casas--Ibarra parameterization}
We split this Autonne--Takagi diagonalization procedure to two steps:
\begin{align}
   &U_1\trans  M_\nu U_1 = \spmat{A& 0 \\ 0&B},\\
   &\spmat{U_2&0\\0&U_3}\trans U_1\trans M_\nu U_1 \spmat{U_2&0\\0&U_3} = \spmat{m\w{diag} & 0 \\ 0 & M\w{diag}},
\end{align}
where $m\w{diag}=\diag(m_1, m_2, m_3)$ and $M\w{diag}=\diag(M_1, M_2)$.
The result of the first step is well-known in series-expanded form:
\begin{align}
U_1 &\simeq \pmat{
  1 - \frac{\vev{\phi_0}^2}{2}\YN^\dagger (MM^*)^{-1}\YN &
 \vev{\phi_0} \YN^\dagger M^{*-1}
\\
-\vev{\phi_0} M^{-1}\YN &
  1 - \frac{\vev{\phi_0}^2}{2} M^{-1}\YN\YN^\dagger M^{*-1}
},\\
 A &\simeq -\vev{\phi_0}^2 \YN\trans M^{-1} \YN,\\
 B &\simeq M + \frac{\vev{\phi_0}^2}{2} \left( \YN\YN^\dagger M^{*-1} + M^{*-1} \YN^* \YN\trans\right)
\end{align}
The second step is expressed by
\begin{align}
 U_2\trans A U_2 &= m\w{diag}, &
 U_3\trans B U_3 &= M\w{diag}.
\end{align}
We also have the expression of the mass eigenstates:
\begin{align}
 \text{lighter}:&\quad
  U_2^\dagger\left[
\nu
-\vev{\phi_0} \YN^\dagger M^{*-1} n
-\frac{\vev{\phi_0}^2}{2} \YN^\dagger (MM^*)^{-1}\YN\nu
\right]\\
 \text{heavier}:&\quad
  U_3^\dagger\left[
n
+\vev{\phi_0} M^{-1}\YN\nu
-\frac{\vev{\phi_0}^2}{2} M^{-1}\YN\YN^\dagger M^{*-1}n
\right]\\
% R'^\dagger(U_{12}^\dagger \nu + U_{22}^\dagger n).
\end{align}

Combining them,
\begin{align}
 m\w{diag} &=  U_2\trans A U_2\\
&= -\vev{\phi_0}^2U_2\trans \YN\trans M^{-1} \YN U_2 + \Order(\epsilon^4)\\
&=-\vev{\phi_0}^2U_2\trans \YN\trans B^{-1} \YN U_2 + \Order(\epsilon^4)\\
&=-\vev{\phi_0}^2U_2\trans \YN\trans U_3 M\w{diag}^{-1}U_3\trans \YN U_2 + \Order(\epsilon^4)
\end{align}
with $\Order(M\epsilon^n)\sim\Order(\vev{\phi_0}^n/M_1^{n-1})$.
Now $R:=-i\vev{\phi_0}M\w{diag}^{-1/2} U_3\trans Y_\nu U_2 m\w{diag}^{-1/2}$ satisfies $R\trans R=1$, which is Casas--Ibarra parameterization in general basis,
\begin{equation}
 Y_\nu = i\vev{\phi_0}^{-1}U_3^* \sqrt{M\w{diag}}R\sqrt{m\w{diag}}U_2^\dagger.
\end{equation}

We have not yet defined the lepton basis.
We can assume that we have been using, from the beginning, the charged lepton mass basis for $L$.
Then, we identify the PMNS matrix ($U_{li}$ in Eq.~(14.1) of PDG2018, where $i$ for mass and $l$ for gauge indices) as
\begin{equation}
 U\w{PMNS} \simeq U_2.
\end{equation}
Similarly, the basis for $\NR$ is such that $U_3 \simeq 1$, which corresponds to
\begin{equation}
 M = B + \Order(\epsilon^2) = U_3^* M\w{diag} U_3^\dagger +\Order(\epsilon^2) = M\w{diag} + \Order(\epsilon^2),
\end{equation}
i.e., the basis in which $M_{ab}\simeq\diag(M_1, M_2)$ with $0<M_1\le M_2$.
These basis choice gives the well-known Casas--Ibarra parameterization,\footnote{
If we took the basis in which $M_{ab}\simeq\diag(-M_1, -M_2)$, then $U_3=\diag(-i, -i)$ and we can remove $i$, but it is less plausible and we put $i$ in the parameterization.
}
\begin{equation}
 Y_\nu = i\vev{\phi_0}^{-1}\sqrt{M\w{diag}}R\sqrt{m\w{diag}}U\w{PMNS}^\dagger,
\end{equation}
where
\begin{equation}\begin{split}
 U\w{PMNS} &= \pmat{
                 c_{12}      c_{13}             &                 s_{12}      c_{13}            &       s_{13} e^{-i\delta} \\
 -s_{12}c_{23} - c_{12}s_{23}s_{13}e^{i\delta}  & +c_{12}c_{23} - s_{12}s_{23}s_{13}e^{i\delta} & s_{23}c_{13}\\
 +s_{12}s_{23} - c_{12}c_{23}s_{13}e^{i\delta}  & -c_{12}s_{23} - s_{12}c_{23}s_{13}e^{i\delta} & c_{23}c_{13}
}
\\&\qquad\qquad
\times\diag(1, e^{i\alpha_{21}/2}, e^{i\alpha_{31}/2}).
\end{split}\end{equation}

\subsection{Higgs potential}
Following [1611.03827],

\begin{equation}
 V = -\mu^2(\phi^\dagger \phi) + \lambda(\phi^\dagger \phi)^2,
\end{equation}
which gives $m_h^2 = 2\mu^2 = 2\lambda v^2$ with $\vev{\phi}=v/\sqrt{2}$.

We calculate the threshold correction to $\mu^2$ and $\lambda$ by matching the effective potential.
The difference of the EFT, in which the heavy neutrinos are integrated out, and the full theory is given by
\TODO{Here $M_a$ is physical mass}
\begin{align}
 \Delta V(\phi;Q)
&= V\w{full}(\phi;Q) - V\w{EFT}(\phi;Q)\\
&= \sum_{a=1,2}\frac{-2}{64\pi^2}M_a(\phi)^4\left(\frac{M_a(\phi)^2}{Q^2}-\frac32\right),
\end{align}
where $Q$ is the matching scale and $M_a$ is the physical masses of the heavier neutrinos.
Expanding $\Delta V$ in terms of $\phi$,
\begin{align}
 \Delta V
&= \text{(const.)} - \Delta \mu^2 |\phi|^2 + \Delta \lambda |\phi|^4 + \Order\left(|\phi|^6\right),
\end{align}
where
\TODO{Now $M_a$ is the diagonal majorana masses...}
\begin{align}
 &\Delta\mu^2= -\sum_{a=1,2}\frac{H_a}{8\pi^2}M_a^2\left(1-\log\frac{M_a^2}{Q^2}\right),
\\
 &\Delta\lambda
=
-\frac{1}{16\pi^2}
\Bigl[f_1 \Tr(Y Y^\dagger Y^*Y\trans)
+f_2\Tr(YY^\dagger YY^\dagger)
+f_3H_1^2+f_4H_2^2
\Bigr];
\end{align}
the coefficients are $H_1 = (YY^\dagger)_{11}$, $H_2 = (YY^\dagger)_{22}$, and
\begin{align*}
 f_1&=\frac{2M_1 M_2}{M_2^2-M_1^2}\log\frac{M_2}{M_1},&
 f_2&=\frac{M_2^2\log(M_2^2/Q^2) - M_1^2\log(M_1^2/Q^2)}{M_2^2 - M_1^2} - 1,\\
 f_3 &= 2-\frac{2M_2\log(M_2/M_1)}{M_2-M_1},&
 f_4 &= 2-\frac{2M_1\log(M_2/M_1)}{M_2-M_1};
\end{align*}
for $M_2\simeq M_1$, they approach to $f_1=1$, $f_2=\log({M_1^2}/{Q^2})$, and $f_3=f_4=0$, which gives
\begin{align}
 \Delta \mu^2
&\simeq -\frac{M_1^2}{8\pi^2}\Tr(YY^\dagger)\left(1-\log\frac{M_1^2}{Q^2}\right),\\
 \Delta \lambda
&\simeq
-\frac{1}{16\pi^2}\Bigl[
 \Tr(Y Y^\dagger Y^*Y\trans) + \Tr(YY^\dagger YY^\dagger)\log\frac{M_1^2}{Q^2}
\Bigr].
\end{align}




\end{document}

